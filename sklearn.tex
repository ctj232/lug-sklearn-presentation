\documentclass{lug}


\title{scikit-learn}
\author{C. Travis Johnson}
\date{UNKNOWN}
\institute{Mines Linux Users Group}

\begin{document}

\section{Introduction}
\subsection{Machine Learning}
\begin{frame}{Machine Learning - What is it really?}
    \begin{itemize}[<+->]
        \item Goal: Extract Knowledge from Data
        \item Sometimes called predictive analysis or statistical learning
        \item Given a large matrix of observations $X$, fit a function $f(x)$ that maps observation $x$ to a response variable $y$
    \end{itemize}
\end{frame}

\begin{frame}{Important Terms}
  \begin{description}
    \item[Classifiers] Algorithms that learn functions to map observations to a \textit{discrete} response. E.g., is this tumor malignant or benign? Is this
      email spam or not?
    \item[Regressors] Algorithms that learn functions to map observations to a \textit{continuous} response. E.g., how much should this house cost?
    \item[Underfitting] The learned function is too simple. ``We barely studied for the exam.''
    \item[Overfitting] The learned function is too complex. ``We memorized all the practice problems, but don't understand the material.''
    \item[Generalization] How well does the learned function extend to new observations?
  \end{description}
\end{frame}

\subsection{Scikit-Learn}
\begin{frame}{Scikit-Learn: Machine Learning in Python}
  \begin{itemize}[<+->]
    \item Provides many machine learning algorithms with a common \texttt{Estimator} interface
    \item Built in helpers for common ML tasks (e.g., \texttt{metrics}, \texttt{preprocessing})
    \item Easily combine algorithms to make a complex pipeline\footnote{Sound familiar?}
    \item Relies heavily on \texttt{numpy} and \texttt{scipy}, often used with \texttt{pandas}
  \end{itemize}
\end{frame}

\section{Supervised Learning}
\subsection{Example: Predicting Breast Cancer with Decision Trees}
\begin{frame}{Learning to Predict Breast Cancer}
\inputminted{python3}{examples/cancer-dt.py}
\end{frame}

\begin{frame}{Evaluating Accuracy of a Model}
\inputminted{python3}{examples/cancer-dt2.py}
\end{frame}

\begin{frame}{Other Supervised Learning Models}
  \begin{itemize}[<+->]
    \item Decision trees are a common first step, because they're easy to interpret and don't require much \texttt{preprocessing}
    \item Decision trees are prone to overfitting, so a good improvement is the \texttt{RandomForest}
    \item Support Vector Machines, Logistic/Linear Regression, and Artificial Neural Networks are commonly the first algorithms studied
    \item See the \texttt{scikit-learn} documentation for a comprehensive guide of available algorithms
  \end{itemize}
\end{frame}

%\begin{frame}{Becoming a ``Data Scientist''}
%  \begin{enumerate}
%    \item Get some data
%    \item Pick an algorithm (or algorithm chain)
%    \item Train the model
%    \item Test generalization ability of trained model
%    \item Good enough? Done. Else, go back to step 2.
%  \end{enumerate}
%  \begin{center}
%  It's that easy.
%  \end{center}
%\end{frame}
%
\section{Unsupervised Learning}
\begin{frame}{Distinction from Supervised Learning}
  \begin{description}
    \item[Supervised Learning] You tell the model what the correct answers are for training examples.
    \item[Unsupervised Learning] You ask the model to extract information from a dataset.
    \item[Unsupervised Clustering] Partition data into similar groups. Example: K-Means Clustering
    \item[Unsupervised Transformations] Create new representations of data. Example: Principal Component Analysis
  \end{description}
\end{frame}

\section{Model Evaluation and Improvement}
\begin{frame}{Choice of Evaluation Metric}
  \begin{itemize}[<+->]
    \item Accuracy is not always the best metric for your system
    \item Plenty of others exist, pick the best for your business costs
    \item Look in the \texttt{sklearn.metrics} module for alternatives
    \item You can also use your own evaluation function!
  \end{itemize}
\end{frame}

\begin{frame}{Cross Validation}
\end{frame}

\begin{frame}{Grid Search with Cross Validation}
\end{frame}

\section{Pipelines}
\begin{frame}{Pipelines}
  Use \texttt{Pipeline} to combine multiple estimators into a single estimator. Two conveniences:
  \begin{enumerate}
    \item Convenience: You only have to call fit and predict once on your data to fit a whole sequence of estimators.
    \item Joint parameter selection: You can grid search over parameters of all estimators in the pipeline at once.
  \end{enumerate}
\end{frame}

\begin{frame}{A Simple Pipeline}
  \inputminted[fontsize=\scriptsize]{python3}{examples/pipeline.py}
\end{frame}

\begin{frame}{Grid Search - Tuning a Complex Pipeline}
  \inputminted[fontsize=\scriptsize]{python3}{examples/gridsearch_pipeline.py}
\end{frame}

\begin{frame}[standout]
    \Huge
    Questions?
\end{frame}
\end{document}
